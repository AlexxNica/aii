\section{edg-installfe\label{edg-installfe}\index{edg-installfe}}


AII command line installation frontend

\subsection*{SYNOPSIS\label{edg-installfe_SYNOPSIS}\index{edg-installfe!SYNOPSIS}}
\begin{verbatim}
 edg-installfe [options] <--boot <hostname|regexp>      |
                          --bootlist <filename>         |
                          --configure <hostname|regexp> |
                          --configurelist <filename>    |
                          --install <hostname|regexp>   |
                          --installlist <filename>      |
                          --remove <hostname|regexp>    |
                          --removeall                   |
                          --removelist <filename> >     |
                          --status <hostname|regexp>    |
                          --statuslist <filename> >
\end{verbatim}
\subsection*{DESCRIPTION\label{edg-installfe_DESCRIPTION}\index{edg-installfe!DESCRIPTION}}


This module provides a command line interface to AII tools to 1.
select if a node has to be installed or not 2. add/update nodes
3. remove nodes.  It receives as input from the user
a lists of nodes and their installation status (to be installed/booted from local disk).
It executes the frontend edg-installfe on all installation servers.



Regular expressions can be used to specify hostnames.



Servers can listed in the configuration file (see below).

\subsection*{COMMANDS\label{edg-installfe_COMMANDS}\index{edg-installfe!COMMANDS}}
\begin{description}

\item[--boot $<$hostname$|$regexp$>$] \mbox{}

Select the boot from local disk for $<$hostname$>$. Perl regular
expressions can be used (e.g node00[1-9], node.*).


\item[--bootlist $<$filename$>$] \mbox{}

Select boot from local disk for hosts listed on $<$filename$>$. Hosts have to
be listed one per line. Lines starting with \# are comment.


\item[--configure $<$hostname$|$regexp$>$] \mbox{}

Configure $<$hostname$>$. Perl regular expressions can be used.


\item[--configurelist $<$filename$>$] \mbox{}

Configure hosts listed on $<$filename$>$. Hosts have to
be listed one per line. Lines starting with \# are comment.


\item[--install $<$hostname$|$regexp$>$] \mbox{}

Select the installation for $<$hostname$>$. Perl regular expressions
can be used.


\item[--installlist $<$filename$>$] \mbox{}

Select installation for the hosts listed on $<$filename$>$. Hosts have to
be listed one per line. Lines starting with \# are comment.


\item[--remove $<$hostname$|$regexp$>$] \mbox{}

Remove the configuration for $<$hostname$>$. Perl regular expressions
can be used.


\item[--removelist $<$filename$>$] \mbox{}

Remove configurations for hosts listed on $<$filename$>$. Hosts have to
be listed one per line. Lines starting with \# are comment.


\item[--removeall] \mbox{}

Remove configurations for *ALL* hosts configured. Useful only in case
of problems/test.


\item[--status $<$hostname$|$regexp$>$] \mbox{}

Report the boot status (boot from local disk/install) for $<$hostname$>$ or
for all hostnames that match the regular expression $<$regexp$>$


\item[--statuslist $<$filename$>$] \mbox{}

Report the boot status (boot from local disk/install) for hosts listed
on $<$filename$>$. Hosts have to be listed one per line. Lines starting
with \# are comment.

\end{description}
\subsection*{OPTIONS\label{edg-installfe_OPTIONS}\index{edg-installfe!OPTIONS}}
\begin{description}

\item[--servers $<$user1@server1 user2@server2] \textbf{...$>$}

Installation servers to be updated remotely via ssh. Use'@'
to select the remote user to use. E.g:

\begin{verbatim}
 servers = john@install-1.asdf.fi
\end{verbatim}


If there are more servers, use ' ' to separate them:

\begin{verbatim}
 servers = john@install-1.asdf.fi john@install-2.asdf.fi
\end{verbatim}


If you are running *all* tools (edg-dhcp, edg-nbp, edg-osinstall)
on *this* machine, use just:

\begin{verbatim}
 server = localhost
\end{verbatim}


They will be executed *directly* (no ssh required). Of course you can mix:

\begin{verbatim}
 servers = localhost john@install-1.asdf.fi
\end{verbatim}


Default value: localhost


\item[--nodhcp] \mbox{}

Run do not update DHCP configuration.


\item[--nonbp] \mbox{}

Do not update NBP configuration.


\item[--noosinstall] \mbox{}

Do not update OS installer configurations.

\end{description}
\subsection*{CONFIGURATION FILE\label{edg-installfe_CONFIGURATION_FILE}\index{edg-installfe!CONFIGURATION FILE}}


Default values of command lines options can be specified in the file
\$EDG\_LOCATION/etc/edg-nbp.conf using syntax:

\begin{verbatim}
 <option> = <value>
\end{verbatim}
\subsection*{AUTHORS\label{edg-installfe_AUTHORS}\index{edg-installfe!AUTHORS}}


Enrico Ferro $<$enrico.ferro@pd.infn.it$>$

