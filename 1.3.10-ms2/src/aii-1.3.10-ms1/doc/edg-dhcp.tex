\section{edg-dhcp\label{edg-dhcp}\index{edg-dhcp}}


Add/remove host entries to an ISC DHCP server.

\subsection*{SYNOPSIS\label{edg-dhcp_SYNOPSIS}\index{edg-dhcp!SYNOPSIS}}
\begin{verbatim}
 edg-dhcp [options] <--configure <hostname> <mac> |
                     --configurelist <filename>   |
                     --remove <hostname>          |
                     --removelist <filename>      |
                     --removeall >
\end{verbatim}
\subsection*{DESCRIPTION\label{edg-dhcp_DESCRIPTION}\index{edg-dhcp!DESCRIPTION}}


edg-dhcp is a command line tool to add/remove nodes specific entries to
a ISC DHCP server. Already existing entries are preserved.
The administrator has to prepare the DHCP server configuration file with
all common network definitions and subnets declarations. The tool
add/remove/update entries to the corresponding subnet and restart the
DHCP server. A backup copy of the configuration file is created
before updating it and restarting the DHCP server.



Command line options override default values in \$EDG\_LOCATION/etc/edg-dhcp.conf.

\subsection*{COMMANDS\label{edg-dhcp_COMMANDS}\index{edg-dhcp!COMMANDS}}
\begin{description}

\item[--configure $<$hostname$>$ $<$mac$>$] \mbox{}

Configure $<$hostname$>$ in the DHCP server with the physical $<$mac$>$ address
(syntax: XX:XX:XX:XX:XX:XX). If the node is present its
configuration is removed and  replaced by the new one.


\item[--tftpserver $<$hostname$>$] \mbox{}

TFTP server (optional). Can be specified only with --configure.


\item[--addoptions $<$text$>$] \mbox{}

Additional DHCP options for the node that will be specified
inside the entry host. Can be specified only with --configure;
they should be specified between quotes, e.g.:
 edg-dhcp --configure node002 --addoptions 'filename loader.bin;'


\item[--configurelist $<$filename$>$] \mbox{}

Configure hosts listed on $<$filename$>$. Hosts have to be listed one per line
with the syntax $<$hostname$>$ $<$mac$>$ [tftpserver] [addoptions], where $<$hostname$>$
and $<$mac$>$ are mandatory. Lines with \# are comment. If a different TFTP server
should not be specified but there are additional options, use a ';'.
Additional options are written exactly as they have to written in DHCP
configuration file. An example:

\begin{verbatim}
 # You can use both : and - in the MAC address
 node1         00:80:45:6F:19:1A
 node2.qwer.fi 00-80-45:6F-19-1B  bootserver
 node3.qwer.fi 00:80:45:6F:19:1C  bootserver.qwer.fi filename "down.bin";
 node3.qwer.fi 00-80-45-6F-19-1D  ;                  filename "down.bin";
\end{verbatim}

\item[--remove $<$hostname$>$] \mbox{}

Remove $<$hostname$>$ from the DHCP server configuration.


\item[--removelist $<$filename$>$] \mbox{}

Remove hosts listed on $<$filename$>$ from the DHCP server. Hosts have to
be listed one per line. Lines with \# are comment.


\item[--removeall] \mbox{}

Remove configurations for *ALL* hosts configured. Useful only in case
of problems/test.

\end{description}
\subsection*{OPTIONS\label{edg-dhcp_OPTIONS}\index{edg-dhcp!OPTIONS}}
\begin{description}

\item[--cfgfile $<$path$>$] \mbox{}

Use the as configuration file $<$path$>$ instead of default
\$EDG\_LOCATION/etc/edg-dhcp.conf


\item[--dhcpconf $<$path$>$] \mbox{}

Configuration file for DHCP server (default: /etc/dhcpd.conf)


\item[--restartcmd $<$command$>$] \mbox{}

Command to be used to restart the server (default: /sbin/service
dhcpd restart). Should be provided between quotes, e.g.
 edg-dhcp --configurelist list --restartcmd '/sbin/mydhcpd --restart'.


\item[--norestart] \mbox{}

Update the configuration file but do not restart the server.

\end{description}
\subsection*{CONFIGURATION FILE\label{edg-dhcp_CONFIGURATION_FILE}\index{edg-dhcp!CONFIGURATION FILE}}


Default values of command lines options can be specified in the file
\$EDG\_LOCATION/etc/edg-dhcp.conf using syntax:

\begin{verbatim}
 <option> = <value>
\end{verbatim}


e.g.:

\begin{verbatim}
 dhcpconf = /etc/my_dhcpd.conf
\end{verbatim}
\subsection*{AUTHORS\label{edg-dhcp_AUTHORS}\index{edg-dhcp!AUTHORS}}


Enrico Ferro $<$enrico.ferro@pd.infn.it$>$

